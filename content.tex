%%%%%%%%%%%%%%%%%%%%%%%%%%%%%%%%%%%%%%%%%%%%%%%%%%%%%%%%%%%%%%%%%%
% Don't remove this line.
\usepackage{kotex}
\defaultfontfeatures{Ligatures=TeX}
\xetexkofontregime{hangul}[]

%%% set font 
\setmonofont{Monaco}
\setsanshangulfont{KoreanHNDB-R}[] %a하늘땅별땅
%\setsanshangulfont{SeoulHangangL}[] %서울한강체L
%\setsanshangulfont{08SeoulNamsanL}[] %서울남산체L
%\setsanshangulfont{HCRDotumLVT}[] % 함초롬돋움 LVT
%\setsanshangulfont{BM-HANNA.ttf}[] % 배달의 민족 한나
%\setsanshangulfont{BM-JUA.ttf}[] % 배달의 민족 주아

\usepackage[export]{adjustbox}
\usepackage{booktabs}

\usepackage{kotex}
\defaultfontfeatures{Ligatures=TeX}
\xetexkofontregime{hangul}[]
\setmonofont{Monaco}
\setsanshangulfont{KoreanHNDB-R}[] %a하늘땅별땅

\usepackage{listings}
\lstset{
    basicstyle=\footnotesize
}

\newcommand\independent{\protect\mathpalette{\protect\independenT}{\perp}}
\def\independenT#1#2{\mathrel{\rlap{$#1#2$}\mkern2mu{#1#2}}}
\usepackage{tikz}
\usetikzlibrary{positioning}
\usetikzlibrary{shapes,plotmarks,decorations.pathreplacing,automata}
\usetikzlibrary{shapes,shapes.multipart}
\makeatletter
\tikzset{circle split part fill/.style  args={#1,#2}{%
 alias=tmp@name, % Jake's idea !!
  postaction={%
    insert path={
     \pgfextra{% 
     \pgfpointdiff{\pgfpointanchor{\pgf@node@name}{center}}%
                  {\pgfpointanchor{\pgf@node@name}{east}}%            
     \pgfmathsetmacro\insiderad{\pgf@x}
      \fill[#1] (\pgf@node@name.base) ([xshift=-\pgflinewidth]\pgf@node@name.east) arc
                          (0:180:\insiderad-\pgflinewidth)--cycle;
      \fill[#2] (\pgf@node@name.base) ([xshift=\pgflinewidth]\pgf@node@name.west)  arc
                           (180:360:\insiderad-\pgflinewidth)--cycle;            %  \end{scope}   
         }}}}}  
\makeatother  
% %\tikzset{every node/.style={circle split, draw,  minimum width=0.6cm}}
% \tikzstyle{both}=[circle split, draw, circle split part fill={black,black}]
% \tikzstyle{neither}=[circle split, draw, circle split part fill={white,white}]
% \tikzstyle{t1}=[circle split, draw, circle split part fill={black,white}]
% \tikzstyle{t2}=[circle split, draw, circle split part fill={white,black}]

%%% Local Variables: 
%%% mode: latex
%%% TeX-master: t
%%% End: 

\usepackage{amsmath}
\makeatletter
\newcommand{\oset}[2]{%
  {\mathop{#2}\limits^{\vbox to -.3\ex@{\kern-\tw@\ex@
   \hbox{\tiny #1}\vss}}}}
\makeatother

%%% Local Variables: 
%%% mode: latex
%%% TeX-master: t
%%% End: 


\newcommand{\whitepic}[2]{
  \begin{center}
    \begin{tikzpicture}[scale=4, show background rectangle, background
      rectangle/.style={fill=white}]
      \node (label) at
      (0,0)[draw=black]{\includegraphics[width=#2\linewidth]{#1}};
    \end{tikzpicture}
  \end{center}
}

\newcommand{\rwhitepic}[2]{
\hfill   \begin{tikzpicture}[scale=4, show background rectangle, background
      rectangle/.style={fill=white}]
      \node (label) at
      (0,0)[draw=black]{\includegraphics[width=#2\linewidth]{#1}};
    \end{tikzpicture}
}

% set handwritten font, necessary packages are loaded in beamerthemeblackboard.sty
% \ECFAugie

\tolerance=1000
% \input{../include/mybeamer-black}
\setbeamercolor{framesource}{fg=gray}
\setbeamerfont{framesource}{size=\footnotesize}
\makeatletter
\renewcommand{\footnoterule}{}
\define@key{beamerfootnote}{nonumber}[true]{\edef\beamer@footarg{0}\def\@makefnmark{}}% have to set a number in \beamer@footarg, then it won't be automatically generated one. setting \@makefnmark to be empty means the number isn't printed. but only use this in a group else it affects following footnotes!
% instead of 'nonumber', could just use 0 as an optional argument to \footnote, but OP reports that keyval complains about this in some situations
% http://tex.stackexchange.com/questions/89539
\newcommand{\br}[1]{{\footnote[nonumber]{%
    \begin{beamercolorbox}[right,wd=\dimexpr\hsize-1.8em\relax]{framesource}
      \setlength\topsep{0pt}\rule{0.65\textwidth}{0.4mm}\vspace{0.5em}\\[-6pt]
      \usebeamerfont{framesource}\usebeamercolor[fg]{framesource}\itshape{#1}
    \end{beamercolorbox}}}}
\setlength{\footnotesep}{0cm}%\footnotesep is the space between footnotes (generated with a \rule)
\makeatother

\institute{\href{http://blog.dokenzy.com}{http://blog.dokenzy.com}}
\usepackage{tikz}\usetikzlibrary{arrows,positioning,shapes.geometric,calc,through,backgrounds}
%\titlegraphic{\includegraphics[height=.6\textheight]{dilstats}}
\usetheme{default}

%%%%%%%%%%%%%%%%%%%%%%%%%%%%%%%%%%%%%%%%%%%%%%%%%%%%%%%%%%%%%%%%%%


\author{Dokenzy}
\date{\vspace{-1em}}
\title{Blackboard 테마 사용하기}
\begin{document}

\maketitle

\begin{frame}[fragile]
\frametitle{사용법}
\begin{enumerate}
\item 이 템플릿을 다운로드하세요.
\item `content.tex' 파일을 수정하세요.
\item 컴파일하세요.
\end{enumerate}
\end{frame}


\begin{frame}[fragile]
\frametitle{1. 다운로드}
두 가지 방법 중 원하는 방법으로 이 템플릿을 다운로드하세요.
\begin{enumerate}
\item \lstinline|git clone https://github.com/dokenzy/talk-outline.git|
\item \href{https://github.com/dokenzy/talk-outline/archive/master.zip}{zip파일 다운로드}
\end{enumerate}
\end{frame}


\begin{frame}[fragile]
\frametitle{2. `content.tex'파일을 수정하세요.}
각 슬라이드는 다음과 같은 형식으로 만들 수 있습니다.
\begin{lstlisting}

 \begin{frame}
 \frametitle{슬라이드 제목}
 \framesubtitle{부제목}
 발표하고 싶은 내용을 쓰세요.
 \end{frame}

\end{lstlisting}
\end{frame}


\begin{frame}[fragile]
\frametitle{2-1. 한글폰트를 바꾸세요}
기본폰트는 아시아폰트에서 무료로 배포하는 a하늘땅별땅입니다. 이 폰트는 재배포가 불가능하니, `폰트통' 이라는 프로그램을 통해 직접 다운로드하거나 다른 폰트를 사용해야 합니다. 다음 예를 참고해서 원하는 폰트를 사용하세요.
\begin{lstlisting}

\setsanshangulfont{KoreanHNDB-R}[] %a하늘땅별땅
\setsanshangulfont{SeoulHangangL}[] %서울한강체L
\setsanshangulfont{08SeoulNamsanL}[] %서울남산체L
\setsanshangulfont{HCRDotumLVT}[] % 함초롬돋움 LVT
\setsanshangulfont{BM-HANNA.ttf}[] % 배달의 민족 한나
\setsanshangulfont{BM-JUA.ttf}[] % 배달의 민족 주아

\end{lstlisting}
시스템에 설치하지 않은 폰트는 폰트 파일 이름으로 사용할 수 있습니다.
\end{frame}


\begin{frame}[fragile]
\frametitle{2-1. 소스코드를 위한 팁}
Beamer에서 소스코드를 그대로 사용하기 위해서는 [fragile] 옵션을 사용해야 합니다. 그리고 컴파일할 때에는 \lstinline|-shell-escape| 옵션을 사용해야 합니다.
\begin{lstlisting}

 \begin{frame}[fragile]
 \frametitle{Hello, Beamer}
 you must use [fragile] option.
 \end{frame}

\end{lstlisting}
\end{frame}


\begin{frame}[fragile]
\frametitle{3. 이제 컴파일합니다.}
다음 중 한 가지 방법으로 컴파일합니다. \lstinline|-shell-escape| 옵션이 적용되었습니다.
\begin{enumerate}
\item \lstinline|make handout|
\item \lstinline|make 4up|
\item \lstinline|make slides|
\item \lstinline|make all|

\end{enumerate}

물론 xelatex 명령으로 직접 컴파일할 수도 있습니다.
\end{frame}


\begin{frame}{감사합니다.}
이 템플릿은 chr1swallace의 \href{https://github.com/chr1swallace/talk-outline}{talk-outline} 저장소에서 fork한 것입니다. chr1swallace에게 감사드립니다. 배경이미지는 \href{https://www.overleaf.com/20113zpkgkf}{overleaf.com}에서 구했습니다. 아시아폰트에서 재밌는 폰트를 많이 배포하는데, 라이센스가 조금 아쉽습니다.
\vfill
\end{frame}

\end{document}

%%% Local Variables: 
%%% mode: latex
%%% TeX-master: "slides"
%%% End: 
